%-----------------------------------------------------------------------

\documentclass[a4paper]{article}

%-----------------------------------------------------------------------
% Knitr.

\usepackage[]{graphicx}
\usepackage[]{color}
% maxwidth is the original width if it is less than linewidth otherwise
% use linewidth (to make sure the graphics do not exceed the margin).
\makeatletter
\def\maxwidth{%
  \ifdim\Gin@nat@width>\linewidth
    \linewidth
  \else
    \Gin@nat@width
  \fi
}
\makeatother

\definecolor{fgcolor}{rgb}{0.345, 0.345, 0.345}
\newcommand{\hlnum}[1]{\textcolor[rgb]{0.686,0.059,0.569}{#1}}%
\newcommand{\hlstr}[1]{\textcolor[rgb]{0.192,0.494,0.8}{#1}}%
\newcommand{\hlcom}[1]{\textcolor[rgb]{0.678,0.584,0.686}{\textit{#1}}}%
\newcommand{\hlopt}[1]{\textcolor[rgb]{0,0,0}{#1}}%
\newcommand{\hlstd}[1]{\textcolor[rgb]{0.345,0.345,0.345}{#1}}%
\newcommand{\hlkwa}[1]{\textcolor[rgb]{0.161,0.373,0.58}{\textbf{#1}}}%
\newcommand{\hlkwb}[1]{\textcolor[rgb]{0.69,0.353,0.396}{#1}}%
\newcommand{\hlkwc}[1]{\textcolor[rgb]{0.333,0.667,0.333}{#1}}%
\newcommand{\hlkwd}[1]{\textcolor[rgb]{0.737,0.353,0.396}{\textbf{#1}}}%

\usepackage{framed}
\makeatletter
\newenvironment{kframe}{%
  \def\at@end@of@kframe{}%
  \ifinner\ifhmode%
  \def\at@end@of@kframe{\end{minipage}}%
\begin{minipage}{\columnwidth}%
  \fi\fi%
  \def\FrameCommand##1{\hskip\@totalleftmargin \hskip-\fboxsep
    \colorbox{shadecolor}{##1}\hskip-\fboxsep
    % There is no \\@totalrightmargin, so:
    \hskip-\linewidth \hskip-\@totalleftmargin \hskip\columnwidth}%
  \MakeFramed {\advance\hsize-\width
    \@totalleftmargin\z@ \linewidth\hsize
    \@setminipage}}%
{\par\unskip\endMakeFramed%
  \at@end@of@kframe}
\makeatother

\definecolor{shadecolor}{rgb}{.97, .97, .97}
\definecolor{messagecolor}{rgb}{0, 0, 0}
\definecolor{warningcolor}{rgb}{1, 0, 1}
\definecolor{errorcolor}{rgb}{1, 0, 0}
\newenvironment{knitrout}{}{}

\usepackage{alltt}

%-----------------------------------------------------------------------
% Pacotes.

\usepackage[utf8]{inputenc}
\usepackage[brazil]{babel}
\usepackage[T1]{fontenc}
\usepackage{eulervm}
\usepackage{palatino}
\usepackage{inconsolata}
\usepackage[hmargin={1.7cm, 1.7cm}, vmargin={1.7cm, 1.7cm}]{geometry}
\usepackage{amsmath, amsfonts, mathrsfs, amssymb, amsthm}
\usepackage{indentfirst, graphicx, calc, setspace, icomma}
\usepackage{wrapfig}
\usepackage[table, usenames, dvipsnames]{xcolor}
\usepackage{colortbl}
\usepackage{float}
\usepackage{multicol, multirow}
\usepackage[hang]{caption}
\usepackage{enumerate}
\usepackage{paralist}

%--------------------------------------------
% Tikz.

\usepackage{tikz}
\usepackage{pgfplots}
\usepackage{pgfplotstable}

\usetikzlibrary{arrows, calc, matrix, positioning, trees,
                decorations.pathmorphing, backgrounds, fit}

%-------------------------------------------
% ProbSoln.

% \usepackage[draft]{probsoln}
\usepackage{probsoln}
\setkeys{probsoln}{fragile} % Para permitir verbatim no \defproblem.

\renewcommand*{\incorrectitemformat}[1]{
  \hspace{\fboxsep}\hspace{\fboxrule}
  #1
  \hspace{\fboxsep}\hspace{\fboxrule}
}

% Nome da solução: solução, resposta, resultado.
\renewcommand{\solutionname}{}
\renewenvironment{solution}{
  \begin{center}\rule{0.7\linewidth}{0.2mm}\end{center}}{}

\newenvironment{mysolution}{
  \begin{solution}
  }{
  \end{solution}
}

%-----------------------------------------------------------------------
% Definições.

% Linhas horizontais.
\newcommand{\HRule}{\noindent\rule{\linewidth}{0.2mm}}

% Configurando um novo estilo de teorema para exercícios.
\newtheoremstyle{exercicio}{3pt}{3pt}{\upshape}{}{}{.}{.5em}{}

%~ {3pt}  % Espaço antes.
%~ {3pt}  % Espaço depois.
%~ {\upshape} % Body font.
%~ {}     % Identação.
%~ {}     % Fonte do título.
%~ {.}    % Pontuação depois do teorema.
%~ {.5em} % Espaço depois do título.
%~ {}

\theoremstyle{exercicio}
\newtheorem{ex}{}[] % {Exercício}[section]

% Lista centralizada.
\newenvironment{clist}{
  \begin{center}\begin{inparaenum}[a)]}{
    \end{inparaenum}\end{center}}

%-----------------------------------------------------------------------

\begin{document}
\pagestyle{empty}

\def\cabecalho{
  \HRule\\
  \noindent
  \begin{minipage}[t]{0.75\textwidth}
    \baselineskip 12pt
    $if(chair)$$chair$$if(class)$ - $class$$endif$\\ $endif$
    $if(lecturer)$ \hfill $url$\\ $endif$
    $if(lecturer)$$lecturer$$if(email)$ ($email$)$endif$\\ $endif$
    $if(department)$$department$\\ $endif$
    $title$ $if(date)$\hfill $date$$endif$\\
    Acadêmico: \hfill grr: \hspace{2cm}
  \end{minipage}
  \hfill
  \begin{minipage}[t]{0.25\textwidth}
    \begin{flushright}
      \vspace*{-0.35cm}
      \includegraphics[height=2cm]{ufpr-logo.png}
    \end{flushright}
  \end{minipage}\\
  \HRule

  \hspace{0.5cm}
}

$if(answers)$
\$answers$answers
$endif$

$for(exercises)$
\loadallproblems[exerc]{$exercises$}
$endfor$

% \foreachproblem[exerc]{
%   \begin{ex}
%     \label{prob:\thisproblemlabel}
%     \thisproblem
%     \HRule
%   \end{ex}
%   \vspace{1ex}
% }

%-----------------------------------------------------------------------
% Se 0, indica a mesma prova para todos os alunos e se 1, indica provas
% diferentes.

\def\mycmd{$if(varied)$$varied$$else$0$endif$}

% Opção inexistente.
\ifx\mycmd\undefined

%-------------------------------------------
% Provas iguais para todos os alunos.

\else
\if\mycmd0
  \cabecalho

  $body$

  \foreachproblem[exerc]{
    \begin{ex}
      \label{prob:\thisproblemlabel}
      \thisproblem
      \HRule
    \end{ex}
    \vspace{1ex}
  }

%-------------------------------------------
% Provas diferentes por aluno.

\else
\if\mycmd1
  \foreachproblem[exerc]{
    \cabecalho

    $body$

    \label{prob:\thisproblemlabel}
    \setcounter{table}{0}
    \setcounter{figure}{0}
    \setcounter{ex}{0}
    \thisproblem
    \vfill
    \vspace{1ex}
    \HRule
    \newpage
  }

\fi
\fi

\end{document}
\endinput
